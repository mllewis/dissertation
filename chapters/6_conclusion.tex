%!TEX root = ../dissertation.tex

\chapter{Conclusion}
\label{chapter:conclusion}
The motivating question for the present body was of work was: What are the causal processes that lead to regularities in language structure?   In answering this question, we appealed to the notion of timescales. The working hypothesis was that pressures at shorter timescales, namely, language use and acquisition, shape language structure at the timescale of language evolution. We hypothesized that these processes unfolded through dynamics between adjacent timescales, such that pressures at the language use timescale, for example, influenced dynamics at the acquisition timescale.  As a case study in exploring this proposal, we focused on a newly discovered regularity in the structure of natural language: a complexity bias.

In Chapter 2, we found that adult participants demonstrated a productive bias to map a long novel word onto a more complex referent in a word mapping task, across a range of stimuli and conditions. We  also found evidence for this bias across 80 natural languages and a range of words. This work suggested that  the bias was not restricted to English and that the notion of conceptual complexity applied to meanings beyond physical objects. In Chapter 3, we conducted a series of studies trying to understand the nature of conceptual complexity. This work was somewhat inconclusive, but provided some support for the possibility that conceptual complexity is related to the number of primitives associated with a word's meaning. In Chapter 4, we sought to understand {\it which} pressures at shorter timescales lead to the emergence of a complexity bias at the language evolution timescale. We explored qualitative predictions of four hypotheses---Efficient Naming, Learning, Memory, and Pragmatic---and found support for the Learning and Memory Hypotheses. Finally, in Chapter 5, we examined a broader space of linguistic phenomenona and environmental factors in the context of language use. Consistent with the Linguistic Niche Hypothesis, we found evidence that languages systematically vary with aspects of the climate in which they are spoken and cognitive characteristics of the population of speakers. While this work is inherently correlational, it provides suggestive evidence of causal link between environments and languages.

One of the initial motivating hypotheses for the present body of work was the idea that a pragmatic pressure at the language use timescale  may have lead to the bias at the language evolution timescale. This hypothesis derived from the observation of a parallel structure between Horn-style asymmetries in phrase length (``started the car" vs.\ ``got the car to start") and the complexity bias in the lexicon, as well as from Information Theory \cite{horn1984,shannon1948}. With this initial motivation, we indeed found evidence that a pragmatic pressure was reflected in the lexicon. Nevertheless, subsequent work, particularly in Chapter 4, challenged the notion that it is a pragmatic pressure  per se at the language use timescale that lead to the bias. This unexpected result highlights the challenges in positing causal accounts for phenomenon over very long timescales, as well as attempting to differentiate hypotheses that make  similar predictions.

Indeed, one of the central limitations of the present body of work is the qualitative nature of the predictions derived from  hypotheses about the origins of the complexity bias. Given that the different hypotheses make very similar predictions, and that the hypotheses are not in principle mutually exclusive with each other, it becomes difficult to distinguish these hypotheses within a qualitative framework. In addition, the fact that these hypotheses are not mutually exclusive means that there may be interactions between different causal pressures. For example, a memory bias to misremember words in a way consistent with a complexity bias may be synergistic with a learning pressure in children. Furthermore, even if the learning and memory pressures are more or less independent from each other, our work leaves open the question of what the relative contribution of these different pressures are to the bias in natural language. That is, both of these pressures could contribute to the emergence of the bias, but to varying degrees, and the present qualitative framework does not allow us to identify these different contributions. Finally, this work does not shed light on the relative weight of conceptual complexity versus other pressures on word length. To fully understand the importance of conceptual complexity on word length, it will be important in future work to quantify how this pressure compares to the other pressures, like spoken frequency \cite{zipf1936}.

Our work also leaves open a number of questions with respect to the origin of several conceptual constructs---namely, conceptual complexity and iconicity. In Chapter 3, we explored multiple hypotheses about the nature of conceptual complexity, with some evidence pointing to the idea that a meaning with more conceptual primitives is encoded by the cognitive system as more conceptually complex. But, this work did not inform our understanding of how a meaning comes to be a conceptual primitive. The answer to this question is likely relevant to understanding the role of conceptual complexity in language acquisition. If conceptual primitives are related to experience in some way, we might expect that children would have different notions of complexity than adults, and this might interact with the emergence of a complexity bias. It is particularly relevant to the possibility that memory biases in adults and a learning bias in children are synergistic: If adults are making memory errors in a way congruent with their coding system of complexity, but not children's, will these changes make it harder for children to learn word meanings? One important area for future work is therefore  to experimentally explore the role of experience in shaping conceptual complexity, particularly the role of labeling itself.  Relatedly, the present body of  work does not provide an explanation about why the cognitive system might have a bias for iconicity. We leave this puzzling and fascinating question for future work.

%This could be explored by using the geon objects used in chapter 2, and asking whether 

Finally, a remaining issue is understanding the source of the variability in the magnitude of the complexity bias cross linguistically. Chapter 4 provides a starting point in answering this question, but there is still much to do in order to fully understand these differences. A particularly interesting avenue for future research is to explore the precise causal mechanism that underlies correlations between population size and linguistic structure. Chapter 4 outlined a number of possibilities, including the amount of variability in linguistic input and the learning constraints of speakers (L1 vs.\ L2 learners). Future work could examine the variability hypothesis directly by examining the physical distribution of speakers within an area that a region is spoken. Do languages spoken in regions with  large population centers differ from those spoken in regions with the population is more evenly distributed? Another potential source of variability in the magnitude of the complexity bias across languages is the age of the language itself. Future work could explore whether languages with longer language evolution spans differ from those that are relatively new. 

In sum, the complexity bias is a robust cross-linguistic regularity that provides a case study for understanding the evolutionary origins of regularities in natural language. As is evident from this and other work related to language evolution, the causal inference problem for the scientist is difficult: There are many competing similar hypotheses, and the long timescale of the causal mechanisms makes it difficult to intervene on experimentally. Indeed, especially because the phenomenon of interest is so difficult to intervene on, it will be important for future work to have precise quantitative predictions in order to differentiate theories. Nevertheless, the contribution of the current work is to identify a novel linguistic regularity, and to  use the case-study method to begin to differentiate among the space of causal forces leading to linguistic regularities.


 


 %Finally, Accounting for variability?
 %age of language
 %distribution of speakers



%Better test of pragmatic possibility
%* either experimetnally
%* or, in natural setting of new community
   
  %Manipulate conceptual complexity
%Future directions


 


%better test of pragmatic possibility, 

   %Manipulate conceptual complexity

%Conceptual compleixty with children

%deal with pragmatic vs. in-the-moment issue -> "language use" timeslcae

%Discussion of limitations    
%- causal inferences are hard, particularly because not mutually exclusive
% - synergistic
%- what's the relative weight of these to forces; memory errors?
%- what are the conceptual primitives?
%- how does this interact with other forces? Also, spoken frequency




%dynamics between pressures

   