%!TEX root = ../dissertation.tex

\chapter{Conclusion}
\label{chapter:conclusion}
The motivating question for the present body was of work was: What are the causal processes that lead to regularities in language structure?   In answering this question, we appealed to the notion of timescales. The working hypothesis was that pressures at shorter timescales, namely, language use and acquisition, shape language structure at the timescale of language evolution. We hypothesize that these processes unfolded through dynamics between adjacent timescales, such that pressures at the language use timescale, for example, influenced dynamics at the acquisition timescale.  As a case study in exploring this proposal, we focused on a newly discovered regularity in the structure of natural language: a complexity bias.

In Chapter 2, we found that adult participants demonstrated a productive bias to map a long novel word onto a more complexity referent in a word mapping task, across a range of stimuli and conditions. We  also found evidence for this bias across 80 natural languages and a range of words. This work suggested that  the bias was not restricted to English and that the notion of conceptual complexity applied to meanings beyond physical objects. In Chapter 3, we conducted a series of studies trying to understand the nature of conceptual complexity. This work was somewhat inconclusive, but provided some support for the possibility that conceptual complexity is related to the number of primitives associated with the meaning. In Chapter 4, we sought to understand {\it which} pressures at shorter timescales lead to the emergence of a complexity bias at the language evolution timescale. We explored qualitative predictions of four hypotheses---Efficient Naming, Learning, Memory, and Pragmatic---and found support for the Learning and Memory Hypotheses. Finally, in Chapter 5, we examined a broader space of linguistic phenomenon and environmental factors in the context of language use. Consistent with the Linguistic Niche Hypothesis, we found evidence that languages systematically vary with aspects of the climate in which they are spoken and cognitive characteristics of the population of speakers. While this work is inherently correlational, it provides suggestive evidence of causal link between languages and their environments.

One of the central motivating ideas for this body of work was the idea that it was a pragmatic pressure at the language use timescale that may have lead to the bias at the timescale of language evolution. This prediction derived from the observation of a parallel structure between Horn-style asymmetries in phrase length (``started the car" vs.\ ``got the car to start") and the complexity bias in the lexicon, as well as from Information Theory. With this initial motivation, we indeed found evidence that a pragmatic pressure was reflected in the lexicon. Nevertheless, subsequent work, particularly in Chapter 4, challenged the notion that it is a pragmatic pressure  per se at the language use timescale that lead to the bias. This unexpected result highlights the challenges in positing causal accounts for phenomenon over very long timescales, as well as attempting to differentiate hypotheses that make  similar predictions.

Indeed, one of the central limitations of the present body of work is the qualitative nature of the predictions derived from  hypotheses about the origins of the complexity bias. Given that the different hypotheses make very similar predictions, and that the hypotheses are not in principle mutually exclusive with each other, it becomes difficult to distinguish these hypotheses within a qualitative framework. In addition, the fact that these hypotheses are not mutually exclusive means that there may be interactions between different causal pressures. For example, a memory bias to misremember words in a way consistent with a complexity bias may be synergistic with a learning pressure in children facilitating word learning for meanings that are consistent with a complexity bias. Furthermore, even if the learning and memory pressures are more or less independent from each other, this work leaves open the question of what the relative contribution of these different pressures are to the bias in natural language. That is, both of these pressures could contribute to the emergence of the bias, but to varying degrees, and the present qualitative framework does not allow us to identify these different contributions. Finally, this work does not shed light on the relative weight of conceptual complexity versus other pressures on word lengths. To fully understand the importance of conceptual complexity on word length, it will be important in future work to quantify how this pressures compares to the other pressures, like spoken frequency \cite{zipf}. 

This work also leaves a number of questions open with respect to the origin of other conceptual constructs---namely, conceptual complexity and iconicity. In Chapter 3, we explored several hypotheses about the nature of conceptual complexity, with some evidence pointing to the idea that a meaning with more conceptual primitives is more conceptually complex. But, this work did not inform our understanding of how a meaning comes to be a conceptual primitive. The answer to this question is likely relevant to understanding the role of conceptual complexity in language acquisition. If conceptual primitives are related to experience in some way, we might expect that children would have different notions of complexity than adults, and this might interact with the emergence of a complexity bias. It is particularly relevant to the possibility that memory biases in adults and a learning bias in children are synergistic: If adults are making memory errors in a way congruent with their coding system of complexity, but not children's, will this in fact make it harder for children to learn word meanings? One important area for future work is therefore  to experimentally explore the role of experience in shaping conceptual complexity, particularly the role of labeling itself in influencing the conceptual complexity of a meaning.  Relatedly, the present body of  work does not provide an explanation about why the cognitive system might have a bias for iconicity. We leave this puzzling and fascinating question for future work.

%This could be explored by using the geon objects used in chapter 2, and asking whether 

Finally, a remaining issue is understanding the reasoning for variability in the complexity bias cross linguistically. Chapter 4 began provides a starting point in answering this question, but there is still much to do in order to fully understand these differences. A particularly interesting avenue for future research is to explore the precise causal mechanism that underlies correlations between population size and linguistic structure. Chapter 4 outlined a number of possibilities, including the amount of variability in linguistic input and the learning constraints of speakers (L1 vs.\ L2 learners). Future work could examine the variability hypothesis directly by examining the physical distribution of speakers within an area that a region is spoken. Do languages spoken in regions with  large populations centers differ from those spoken in regions with the population is more evenly distributed. Another pontential source of variability in the magnitude of the complexity bias across langauges is the age of the language itself. Future work could explore whether languages with longer language evolution spans differ from those that are relatively new. 

Do the different hypotheses make different quantitative predictions (predicitons about effect sizes?) Arugably, because the phenomeonon is so difficult to intervene on causal, it becomes more important to have quantitiative, precise predicitons to differentiate theories.

 


 Finally, Accounting for variability?
 age of language
 distribution of speakers



Better test of pragmatic possibility
* either experimetnally
* or, in natural setting of new community
   
  Manipulate conceptual complexity
Future directions


 


better test of pragmatic possibility, 

   Manipulate conceptual complexity

Conceptual compleixty with children

%deal with pragmatic vs. in-the-moment issue -> "language use" timeslcae

%Discussion of limitations    
%- causal inferences are hard, particularly because not mutually exclusive
% - synergistic
%- what's the relative weight of these to forces; memory errors?
%- what are the conceptual primitives?
%- how does this interact with other forces? Also, spoken frequency




dynamics between pressures

   