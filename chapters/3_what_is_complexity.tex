%!TEX root = ../dissertation.tex

\chapter{Introduction}
\label{chapter:introduction}

\cite{laurence1999concepts,anaki2009familiarity,feldman2016simplicity,goodman2008rational,haskell2011linguistic,lupyan2008conceptual,feldman2000minimization}


\bgroup
\def\arraystretch{1.5}% 

\begin{table}[t]
\footnotesize
\centering
\begin{tabular}{l l l l l }
 \toprule
 \textbf{Theory} &  \textbf{\begin{tabular}[c]{@{}l@{}}Relevant\\Dimension\end{tabular}} & \textbf{Prediction}& \textbf{\begin{tabular}[c]{@{}l@{}}Relevant\\Studies\end{tabular}} \\
 \toprule
Classical theory & \# of primitives & \multicolumn{1}{p{5cm}}{Concepts with longer definitions (and thus more primitives) will be more complex.} & Studies 1-3 \\
Exemplar theory  &  \# of exemplars & \multicolumn{1}{p{5cm}}{Concepts with more  exemplars will be more complex.} & Studies 4-5  \\
Prototype  & \# of exemplars & \multicolumn{1}{p{5cm}}{Concepts with fewer exemplars will have more uncertainty, and thus be more complex.} & Studies 4-5    \\
Prototype  & Amount of variability &\multicolumn{1}{p{5cm}}{ Concepts with more variability will be more complex.} & Study 6 \\
Semantic network  &  Entropy of distance to associates &\multicolumn{1}{p{5cm}}{ Concepts with more uniform distribution  of distance to associates will be more complex.} & Study 7 \\

 \bottomrule
\end{tabular}
\caption{Summary of studies.}
\label{complexity_pred_summary_table}
\end{table}
\egroup

\section{Experiment 1: Descriptions of objects}
\subsection{Methods}
\subsubsection{Participants} 
\subsubsection{Stimuli} 
\subsubsection{Procedure}
\subsection{Results and Discussion}

\section{Experiment 2a: Definition complexity norming }
\subsection{Methods}
\subsubsection{Participants} 
\subsubsection{Stimuli} 
\subsubsection{Procedure}
\subsection{Results and Discussion}

\section{Experiment 2b: Definition word mapping}
\subsection{Methods}
\subsubsection{Participants} 
\subsubsection{Stimuli} 
\subsubsection{Procedure}
\subsection{Results and Discussion}

\section{Study 3: McCrae feature norms }
\subsection{Methods}
\subsection{Results and Discussion}

\section{Experiment 4a: Simultaneous frequency}
\subsection{Methods}
\subsubsection{Participants} 
\subsubsection{Stimuli} 
\subsubsection{Procedure}
\subsection{Results and Discussion}

\section{Experiment 4b: Sequential frequency}
\subsection{Methods}
\subsubsection{Participants} 
\subsubsection{Stimuli} 
\subsubsection{Procedure}
\subsection{Results and Discussion}

\section{Experiment 5: Facts}
\subsection{Methods}
\subsubsection{Participants} 
\subsubsection{Stimuli} 
\subsubsection{Procedure}
\subsection{Results and Discussion}

\section{Experiment 6: Concept variability}
\subsection{Methods}
\subsubsection{Participants} 
\subsubsection{Stimuli} 
\subsubsection{Procedure}
\subsection{Results and Discussion}

\section{Study 7: Entropy of associates}
\subsection{Methods}
\subsubsection{Participants} 
\subsubsection{Stimuli} 
\subsubsection{Procedure}
\subsection{Results and Discussion}






